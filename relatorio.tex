\documentclass{article}
\usepackage{hyphenat}
\usepackage{amssymb}

\title{Trabalho EBD2 - segunda unidade}
\author{Eduardo Marinho}
\date{abril 2024}

\begin{document}

\maketitle

\section{Resumo}
\leavevmode
\indent Esse relatório tem como objetivo explicar a implementação da ferramenta para 
auxílio do monitoramento do parque das dunas, a qual foi realizada em C++ utilizando 
uma AVL-tree para armazenar os valores.

\section{Introdução}
\subsection{Cenário}
\leavevmode
\indent TODO
\subsection{Estrutura da árvore}
\leavevmode
\indent A árvore utilizada foi a AVL-tree. Um tipo de árvore binária balanceada, ou seja,
com a garantia de que, para todos os nodos da árvore, a diferença da altura da subarvore
da esquerda para a altura da subarvore da direita não é maior que 1, fazendo com que a
altura máxima da árvore seja log(n). Para manter essa propriedade a AVL-tree realiza
rotações quando um node não estiver balanceado, de modo a torná-lo balanceado.

\section{Modelagem}
\leavevmode
\indent Foi criada uma classe "Dados" para encapsular todos os dados da execução, e implementar
as funções auxiliares para ler e escrever os dados no arquivo, assim como as funções
para consultar e alterar os dados, dentre elas: "incluir animal" para incluir um novo 
animal no banco de dados, "remover animal" para remover um animal, "consultar fauna"
para consultar as infomações de um animal específico, "inserir monitoramento" para 
inserir dados de monitoramento para um animal específico, entre outras. \\
\indent Além disso, foi criado structs auxiliares como DadosDoAnimal e DadosDeMonitoramento,
para encapsular os respectivos dados e métodos de entrada e saida dos valores relacionados.\\
\indent Ao início da execução o arquivo é lido pela class "Dados", onde todos os dados são
armazenados e então, é criado um loop, onde é lido a operação que o usúario deseja
realizar e os dados necessários, caso hajam, o loop acaba quando o usúario escolher
a opção de acabar a execução, fazendo com que o loop acabe e os dados sejam salvos
no mesmo arquivo de entrada.

\section{Algoritmos}
Considere que o nodo contém a seguinte estrutura: \\
struct nodo \{ \\
\indent int chave; \\
\indent int fatorDeBalanceamento; \\
\indent nodo filhoDaEsquerda; \\
\indent nodo filhoDaDireita; \\
\}; \\
\subsection{Inserção}
\subsubsection{Complexidade}
\leavevmode
\indent Para realizar a inserção em uma AVL tree é necessário encontrar o local a qual se 
deve inserir o nodo, para isso deve percorrer um número de nodos equivalente a O(log(n)), 
onde n é o número de nodos da arvore. Ademais, sabe-se que as etapas auxiliares no algoritmo, 
como rotacionar a arvore para balanceá-la e colocar o nodo no local tem complexidade constantes.
Portanto, a complexidade do algoritmo de inserção é O(log(n)).
\subsubsection{Pseudocódigo}
Inserir(raiz, chave) { \\
\indent if raiz == null \{ \\
\indent\indent raiz = new Nodo \\
\indent\indent raiz.chave = chave \\
\indent\indent retornar \\
\indent \} \\
\indent   if $valor < raiz.valor$ \{ \\
\indent\indent raiz.esquerda = InserirAVL(raiz.esquerda, valor) \\
\indent\indent --raiz.fatorDeBalanceamento; \\
\indent} else if $valor > raiz.valor$ \{ \\
\indent\indent raiz.direita = InserirAVL(raiz.direita, valor) \\
\indent\indent ++raiz.fatorDeBalanceamento; \\
\indent \} else \{ \\
\indent\indent retornar raiz // Valor já esta inserido \\
\indent \} \\
\indent if (raiz.fatorDeBalanceamento == 2) \{ \\
\indent\indent rotaçãoEsquerda(raiz) \\
\indent \} else if (raiz.fatorDeBalanceamento == -2) { \\
\indent\indent rotaçãoDireita(raiz) \\
\indent \} \\
\}

RotacaoDireita(raiz) \{ \\
\indent novaRaiz = raiz.filhoDaEsquerda; \\
\indent if (raiz == raiz.pai.filhoDaEsquerda) \{ \\
\indent\indent raiz.pai.filhoDaEsquerda = novaRaiz; \\
\indent \} else \{ \\
\indent\indent raiz.pai.filhoDaDireita = novaRaiz; \\
\indent \} \\ 
\indent novaRaiz.pai = raiz.pai; \\
\indent raiz.pai = novaRaiz; \\
\indent raiz.filhoDaEsquerda = novaRaiz.filhoDaDireita; \\
\indent novaRaiz.filhoDaDireita = raiz; \\
\indent if (raiz.filhoDaEsquerda != null) \{ \\
\indent\indent raiz.filhoDaEsquerda.pai = raiz; \\
\indent \} \\
\} 

RotacaoEsquerda(raiz) \{ \\
\indent novaRaiz = raiz.filhoDaDireita; \\
\indent if (raiz == raiz.pai.filhoDaEsquerda) \{ \\
\indent\indent raiz.pai.filhoDaEsquerda = novaRaiz; \\
\indent \} else \{ \\
\indent\indent raiz.pai.filhoDaDireita = novaRaiz; \\
\indent \} \\
\indent novaRaiz.pai = raiz.pai; \\ 
\indent raiz.pai = novaRaiz; \\
\indent raiz.filhoDaDireita = novaRaiz.filhoDaEsquerda; \\
\indent novaRaiz.filhoDaEsquerda = raiz; \\
\indent if (raiz.filhoDaDireita != null) \{ \\
\indent\indent raiz.filhoDaDireita.pai = raiz; \\
\indent    \} \\
\} \\

\subsection{Procura} 
\subsubsection{Complexidade}
\leavevmode
\indent Como o máximo de operações realizadas é equivalente a altura da arvore e considerando 
que a altura maxima da arvore é log(n), tendo em vista que ela é balanceada, a complexidade 
do algoritmo é O(log(n)).
\subsubsection{Pseudocódigo}
procurar(alvo) \{ \\
\indent atual = raiz; \\
\indent while (atual != null) \{ \\

\indent\indent if ($atual.chave > alvo$) \{ \\
\indent\indent\indent atual = atual.filhoDaEsqueda; \\
\indent\indent \} else if ($atual.chave < alvo$) \{ \\
\indent\indent\indent atual = atual.filhoDaDireita; \\
\indent\indent \} else \{ \\
\indent\indent\indent retorne atual; \\
\indent\indent \} \\
\indent\} \\ 
\indent returne atual; \\
\} \\

\subsection{Remocão} 
\subsubsection{Complexidade}
\leavevmode
\indent Como é preciso realizar o algoritmo de procura para achar o nodo a ser removido, 
e como as operações de remoção e de rotação da árvore tem complexidades constantes
o algoritmo de remoção tem complexidade de O(log(n)).
\subsubsection{Pseudocódigo}
\leavevmode
\indent TODO

\section{Conclusão}
\leavevmode
\indent TODO

\end{document}
